\documentclass[sigconf]{acmart}
\usepackage{booktabs} % For formal tables


% Copyright
%\setcopyright{none}
%\setcopyright{acmcopyright}
%\setcopyright{acmlicensed}
\setcopyright{rightsretained}
%\setcopyright{usgov}
%\setcopyright{usgovmixed}
%\setcopyright{cagov}
%\setcopyright{cagovmixed}


% DOI
\acmDOI{10.475/123_4}

% ISBN
\acmISBN{123-4567-24-567/08/06}

%Conference
\acmConference[CIKM'17]{ACM CIKM}{Nov 2017}{Singapore } 
\acmYear{2017}
\copyrightyear{2016}
\acmPrice{15.00}


\begin{document}
\title{Academic Concept Extractor (ACE): Unsupervised concept based entity extraction from scientific titles}
%\titlenote{Produces the permission block, and
 % copyright information}
%\subtitle{Extended Abstract}
%\subtitlenote{The full version of the author's guide is available as
 % \texttt{acmart.pdf} document}


\author{Adit Krishnan, Aravind Sankar, Shi Zhi, Jiawei Han}
%\authornote{Dr.~Trovato insisted his name be first.}
%\orcid{1234-5678-9012}
\affiliation{%
  \institution{Department of Computer Science}
  \streetaddress{University of Illinois at Urbana-Champaign}
}
\email{{aditk2, asankar3, shizhi2, hanj}@illinois.edu}
%\author{Author 2}
%%\authornote{The secretary disavows any knowledge of this author's actions.}
%\affiliation{%
%  \institution{Institute for Clarity in Documentation}
%  \streetaddress{P.O. Box 1212}
%  \city{Dublin} 
%  \state{Ohio} 
%  \postcode{43017-6221}
%}
%\email{webmaster@marysville-ohio.com}
%\author{Author 3}
%%\authornote{This author is the
% % one who did all the really hard work.}
%\affiliation{%
%  \institution{The Th{\o}rv{\"a}ld Group}
%  \streetaddress{1 Th{\o}rv{\"a}ld Circle}
%  \city{Hekla} 
%  \country{Iceland}}
%\email{larst@affiliation.org}
%\author{Author 4}
%\affiliation{
%  \institution{Brookhaven Laboratories}
%  \streetaddress{P.O. Box 5000}}
%\email{lleipuner@researchlabs.org}
%
%\author{Sean Fogarty}
%\affiliation{%
%  \institution{NASA Ames Research Center}
%  \city{Moffett Field}
%  \state{California} 
%  \postcode{94035}}
%\email{fogartys@amesres.org}
%
%\author{Charles Palmer}
%\affiliation{%
%  \institution{Palmer Research Laboratories}
%  \streetaddress{8600 Datapoint Drive}
%  \city{San Antonio}
%  \state{Texas} 
%  \postcode{78229}}
%\email{cpalmer@prl.com}
%
%\author{John Smith}
%\affiliation{\institution{The Th{\o}rv{\"a}ld Group}}
%\email{jsmith@affiliation.org}
%
%\author{Julius P.~Kumquat}
%\affiliation{\institution{The Kumquat Consortium}}
%\email{jpkumquat@consortium.net}

% The default list of authors is too long for headers}
\renewcommand{\shortauthors}{Author 1 et al.}


\begin{abstract}
This paper studies the extraction and typing of entities from titles of academic literature, in order to gain a deeper understanding of their specific contributions and automate the construction of a problem-solution database to analyze the relations between them. To achieve this goal, we propose an unsupervised, domain independent, two phase algorithm to extract entity mentions and type them into appropriate concepts. In the first phase of our algorithm we propose a generative model which exploits textual and syntactic features to broadly segment titles and type them into concepts. In the second phase, we propose an unsupervised approach based on adaptor grammars to extract fine grained entities of interest without the need for any external resources or human effort, in a purely data driven manner. We analyze literature from diverse scientific domains and show significant gains over state-of-the-art techniques. We also present an analysis and summarization of the knowledge base constructed as part of our algorithm.
\end{abstract}

%
% The code below should be generated by the tool at
% http://dl.acm.org/ccs.cfm
% Please copy and paste the code instead of the example below. 
%
\begin{CCSXML}
<ccs2012>
 <concept>
  <concept_id>10010520.10010553.10010562</concept_id>
  <concept_desc>Computer systems organization~Embedded systems</concept_desc>
  <concept_significance>500</concept_significance>
 </concept>
 <concept>
  <concept_id>10010520.10010575.10010755</concept_id>
  <concept_desc>Computer systems organization~Redundancy</concept_desc>
  <concept_significance>300</concept_significance>
 </concept>
 <concept>
  <concept_id>10010520.10010553.10010554</concept_id>
  <concept_desc>Computer systems organization~Robotics</concept_desc>
  <concept_significance>100</concept_significance>
 </concept>
 <concept>
  <concept_id>10003033.10003083.10003095</concept_id>
  <concept_desc>Networks~Network reliability</concept_desc>
  <concept_significance>100</concept_significance>
 </concept>
</ccs2012>  
\end{CCSXML}

\ccsdesc[500]{Computer systems organization~Embedded systems}
\ccsdesc[300]{Computer systems organization~Redundancy}
\ccsdesc{Computer systems organization~Robotics}
\ccsdesc[100]{Networks~Network reliability}

% We no longer use \terms command
%\terms{Theory}

\keywords{ACM proceedings, \LaTeX, text tagging}


\maketitle

\section{Introduction}

%
%The pre-defined theorem-like constructs are \textbf{theorem},
%\textbf{conjecture}, \textbf{proposition}, \textbf{lemma} and
%\textbf{corollary}.  The pre-defined de\-fi\-ni\-ti\-on-like constructs are
%\textbf{example} and \textbf{definition}.  You can add your own
%constructs using the \textsl{amsthm} interface~\cite{Amsthm15}.  The
%styles used in the \verb|\theoremstyle| command are \textbf{acmplain}
%and \textbf{acmdefinition}.


\section{Conclusions}

%\end{document}  % This is where a 'short' article might terminate

\appendix

\section{Headings in Appendices}
The rules about hierarchical headings discussed above for
the body of the article are different in the appendices.
In the \textbf{appendix} environment, the command
\textbf{section} is used to
indicate the start of each Appendix, with alphabetic order
designation (i.e., the first is A, the second B, etc.) and
a title (if you include one).  So, if you need
hierarchical structure
\textit{within} an Appendix, start with \textbf{subsection} as the
highest level. Here is an outline of the body of this
document in Appendix-appropriate form:

\begin{acks}
  The authors would like to thank ...
\end{acks}

\bibliographystyle{ACM-Reference-Format}
\bibliography{draft} 

\end{document}
